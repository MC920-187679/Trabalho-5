\section{Transformações} \label{sec:transformacoes}

As transformações de imagem foram feitas por operações lineares em coordenadas homogêneas dos pixels. Inicialmente, as operações $\vec{T}$ são aplicadas nos pontos extremos $p_1 = (0, 0, 1)$, $p_2 = (0, H, 1)$, $p_3 = (W, 0, 1)$ e $p_4 = (W, H, 1)$ da imagem $I: H \times W$, de forma que os resultados $p'_i = T \cdot p_i$ pode ser usado para extrair os extremos da imagem transformada. Note que isso é válido, pois as transformações deste trabalho são colineações, isto é, elas mantém a colinearidade entre pontos. Assim, a caixa delimitadora \autocite{bbox} será dada por $(x_{\min}, y_{\min})$ e $(x_{\max}, y_{\max})$, sendo $x_{\min} = \min\left\{ \frac{x_i}{w_i} \;\middle|\; p'_i = (x_i, y_i, w_i)\right\}$ e assim por diante.

Usando a caixa delimitadora como a imagem resultante $I': H' \times W'$, temos os índices $y_{\min} \leq i \leq y_{\max}$ e $x_{\min} \leq j \leq x_{\max}$ como parte da imagem. Assim, podemos aplicar a operação inversa $T'$ para descobrir qual o ponto equivalente na imagem original $I$. A última etapa é a interpolação com os valores discretos da imagem, discutida na \cref{sec:interp} a seguir.

É importante notar, entretanto, que os pixels foram considerados pelo centro. Então, o pixel $ij$ é entendido como $z_{ij} = f(j + 1/2, i + 1/2)$. Para efeito prático, isso feito por uma translação de $T_x = T_y = 1/2$ antes da transformação e outra $T_x = T_y = -1/2$ ao final. Isso faz com que as operações tenham o comportamento esperado.

A seguir estão apresentadas as transformações que podem ser aplicadas com a ferramenta. As matrizes foram baseadas na bibliografia da disciplina \autocite{helio}, com algumas modificações para que a caixa delimitadora sempre comece na origem do plano cartesiano, ou seja, $x_{\min} = y_{\min} = 0$.

Além disso, as coordenadas homogêneas $C$ foram representadas no código-fonte por um tensor $3 \times H \times W$ em que $C_{1ij} = x_{ij} = j$, $C_{2ij} = y_{ij} = i$ e $C_{3ij} = w_{ij} = 1$. Assim, a aplicação da transformação $\vec{T}$ é feita pelo produto interno:
\[
    C'_{ijl} = \sum_{k = 1}^3 \vec{T}^{ik} C_{kjl}
\]

\subsection{Rotação no Plano XY}

    Esse tipo de rotação (em torno do eixo Z, perpendicular à imagem) pode ser feita por:
    \[
        \vec{R} = \begin{bmatrix}
            \cos\alpha & -\sin\alpha & 0 \\
            \sin\alpha & \cos\alpha & 0 \\
            0 & 0 & 1
        \end{bmatrix}
    \]

    No programa, o ângulo $\alpha$ é tratado pelo oposto $-\alpha$, já que o eixo Y é invertido entre as representações matricial e cartesiana. Além disso, a rotação é combinada com uma translação $\vec{L}$ que corrige a posição da caixa delimitadora, sendo:
    \[
        \vec{L} = \begin{bmatrix}
            1 & 0 & -x_{\min} \\
            0 & 1 & -y_{\min} \\
            0 & 0 & 1
        \end{bmatrix}
    \]

\subsection{Escalonamento}

    Sendo $S_x$ e $S_y$ as escalas em X e Y, respectivamente, a matriz de mudança de escala é dada por:
    \[
        \vec{S} = \begin{bmatrix}
            S_x & 0 & 0 \\
            0 & S_y & 0 \\
            0 & 0 & 1
        \end{bmatrix}
    \]

    Para a opção de escala da entrada padrão, temos que $S_x = S_y$. No entanto, para o redimensionamento de $H_i \times W_i$ para $H_f \times W_f$, podemos usar $S_x = W_f / W_i$ e $S_y = H_f / H_i$. Note que operações de escala não precisam de translações corretivas.

    Além disso, uma última operação de escala é feita após a transformação principal. Ela serve para arredondar as dimensões da imagem resultante $H \times W$ para os inteiros positivos mais próximos $H' \times W'$. Desse modo, $H' \geq 1$ e $H' - H \leq 1$, com restrições similares para $W'$.

\subsection{Rotação em Torno do Eixo Y} \label{sec:rotacaoXY}

    Nessa transformação, a imagem é considerada na posição $Z = 0$ e é então rotacionada em torno de Y. O resultado disso é então projetado novamente no plano da imagem. Para isso, é preciso gerar o novo eixo, que é feito com a matriz $\vec{G}$, para podermos rotacionar por $\beta$ com $\vec{R}$. Assim:
    \[
        \vec{G} = \begin{bmatrix}
            1 & 0 & 0 \\
            0 & 1 & 0 \\
            0 & 0 & 0 \\
            0 & 0 & 1
        \end{bmatrix}
        \qquad \qquad
        \vec{R} = \begin{bmatrix}
            \cos\beta & 0 & \sin\beta & 0 \\
            0 & 1 & 0 & 0 \\
            -\sin\beta & 0 & \cos\beta & 0 \\
            0 & 0 & 0 & 1
        \end{bmatrix}
    \]

    A projeção perspectiva $\vec{P}$ é feita considerando o foco em $f = -1$ e a imagem em $Z = D$, que deve ser transladada com $\vec{D}$ para essa posição.
    \[
        \vec{D} = \begin{bmatrix}
            1 & 0 & 0 & 0 \\
            0 & 1 & 0 & 0 \\
            0 & 0 & 1 & D\\
            0 & 0 & 0 & 1
        \end{bmatrix}
        \qquad \qquad
        \vec{P} = \begin{bmatrix}
            1 & 0 & 0 & 0 \\
            0 & 1 & 0 & 0 \\
            0 & 0 & -\frac{1}{f} & 1
        \end{bmatrix} = \begin{bmatrix}
            1 & 0 & 0 & 0 \\
            0 & 1 & 0 & 0 \\
            0 & 0 & 1 & 1
        \end{bmatrix}
    \]

    Além disso, temos as correções de escala. A primeira é uma correção da escala da projeção $\vec{E}$ com fatores $S_x = S_y = D - f = D + 1$. Isso vale, pois projeçoes são homografias, isto é, preserva razões cruzadas. Temos também uma normalização $\vec{N}$, resposável por reduzir a imagem $H \times W$ para um quadrado $1 \times 1$, para podermos usar $D = 2$ como um padrão razoável. A normalização também faz uma translação de $T_x = T_y = -1/2$ para centralizar a imagem na origem.
    \[
        \vec{E} = \begin{bmatrix}
            D + 1 & 0 & 0 \\
            0 & D + 1 & 0 \\
            0 & 0 & 1
        \end{bmatrix}
        \qquad \qquad
        \vec{N} = \begin{bmatrix}
            1 & 0 & -\frac{1}{2} \\
            0 & 1 & -\frac{1}{2} \\
            0 & 0 & 1
        \end{bmatrix} \cdot \begin{bmatrix}
            \frac{1}{W} & 0 & 0 \\
            0 & \frac{1}{H} & 0 \\
            0 & 0 & 1
        \end{bmatrix}
    \]

    Assim, a transformação final é dada $\vec{B} = \vec{N}^{-1} \cdot \vec{E} \cdot \vec{P} \cdot \vec{D} \cdot \vec{R} \cdot \vec{G}$. Essa transformação, assim como na rotação anterior (\cref{sec:rotacaoXY}), necessita de correção da posição da caixa delimitadora, feito com uma matriz similar à $\vec{L}$ naquela situação.
