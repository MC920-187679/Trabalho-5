\section{O Programa} \label{sec:programa}

% TODO: testar 3.6+

A ferramenta foi desenvolvida e testada para as versões de Python 3.6 ou superior. Também foram utilizados os pacotes OpenCV, para entrada e saída de imagens, Numpy, para operações vetorizadas com a imagem, e Matplotlib, para opções de cores para pixels inexistentes.

\subsection{Código-Fonte}

    Neste trabalho foi elaborada a ferramenta \texttt{trasforma.py} que faz a transformação da imagem e aplicação da interpolação. O código responsável pelas operações podem ser encontrados na pasta \texttt{lib}, como apresentado a seguir.

    \begin{description}

        \item[trasnforma.py] Ferramenta de transformação da imagem.

        \item[lib] Conjunto de arquivos com implementações para cada funcionalidade da ferramenta.

        \begin{description}[leftmargin=0\parindent,labelindent=0\parindent]

            \item[idx.py] Criação, transformação e acesso com as coordenadas de cada pixel da imagem.

            \item[linop.py] Operações lineares puras de escalonamento, rotação e translação.

            \item[opimg.py] Operações lineares mais aplicadas a imagem, sempre ajustando o resultado para a origem e retornando as novas dimensões da imagem.

            \item[interp.py] Recuperação da imagem transformada por interpolação dos pontos.

            \item[args.py] Processamento dos argumentos da linha de comando.

            \item[inout.py] Tratamento de entrada e saída de imagens do programa.

            \item[tipos.py] Tipos para checagem estática.
        \end{description}
    \end{description}

    Todas as imagens base para o processamento discutido ao longo do texto estão presente na pasta \texttt{imagens}. Também existe o \textit{script} \texttt{run.sh} em Bash que refaz todos resultados apresentados neste relatório.

\subsection{Execução} % TODO: separar

    A execução de ambos os programas deverá ser feita através do interpretador de Python 3.6+. Os exemplos de execuções a seguir funcionam apenas em Python 3.7+, devido à ordem com que os argumentos são interpretados. No entanto, o \textit{script} \texttt{run.sh} também funciona na versão 3.6.

    O único argumento obrigatório é o caminho da imagem de entrada, preferencialmente PNG, que deverá ser transformada. A imagem de saída é, por padrão, exibida em uma nova janela gráfica, mas pode ser salva em um arquivo com a opção \mintinline{bash}{--output SAIDA} ou \mintinline{bash}{-o SAIDA}.

    A primeira transformação que pode ser feita na imagem é rotação no plano $XY$ por um ângulo $\alpha$. Isso pode ser realizado com \mintinline{bash}{--angulo ALFA} ou \mintinline{bash}{-a ALFA}. Também pode ser feita uma rotação em torno do eixo $Y$, que é projetada em perspectiva para o plano original da imagem. A \textit{flag} para isso é \mintinline{bash}{--beta BETA} ou \mintinline{bash}{-b BETA}. Ambas opções tratam apenas de ângulos em graus.

    % TODO: math eval

    Também temos as opções de escalonamento da imagem. % TODO

    \begin{minted}{bash}
        $ echo MC920 | python3 codificar.py imagens/baboon.png -o saida.png
    \end{minted}

    % TODO: cor

    Todas as opções anteriores estão explicadas com o texto de ajuda da ferramenta, que pode ser acessado com a \textit{flag} \mintinline{bash}{--help} ou apenas \mintinline{bash}{-h}.
